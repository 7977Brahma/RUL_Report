\newpage
\thispagestyle{empty}

\begin{center}
\textbf{\LARGE{ABSTRACT}}
\vspace{.1in}\hrule
\vspace{.05in}\hrule
\vspace{.4in}
\end{center}

\begin{justifying}
Understanding the health and remaining useful life of batteries is vital for effective battery management. Recent advances in machine learning have led to new methods for accurately estimating the State of Health (SOH) and Remaining Useful Life (RUL) of batteries. This report introduces a method to predict battery aging by considering factors such as state of charge, discharge voltage characteristics, internal resistance, and the capacity of a Li-Ion 18650 cell. Various statistical models were deployed on a hardware platform to identify the most effective machine-learning techniques for SOH and RUL estimation. The experimental results show that a deep neural network can predict SOH with a 5\% error margin, while a long short-term memory neural network can estimate RUL with an accuracy of ±10 cycles. This method highlights the potential of machine learning models in real-time applications, enabling optimal battery life management. 

\vspace{0.4in}

Index Terms: Battery management, State of Health (SOH), Remaining Useful Life (RUL), machine learning, Li-Ion 18650 cell, state of charge, discharge voltage characteristics, internal resistance, capacity, deep neural network, long short-term memory neural network, statistical models, real-time applications.
\end{justifying}


\begin{center}
\vspace{2.8in}
Department of Computer Science and Engineering\\
\textbf{National Institute of Technology Silchar, Assam}
\end{center}
