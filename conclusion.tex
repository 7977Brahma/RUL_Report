% conclusion.tex
\chapter{Conclusion}

The primary objective of this report was to estimate the Remaining Useful Life (RUL) of Li-Ion 
batteries using various machine learning models, including Linear Regression, K-Nearest 
Neighbors (KNN), Support Vector Machine (SVM), Decision Tree, and Random Forest. The 
analysis determined that the Random Forest model outperformed the others, achieving the 
highest R² score of 0.9608, indicating its superior accuracy and robustness in capturing 
complex interactions and non-linear relationships between features. Critical variables such as 
discharge time, time at specific voltages, charging time, and cumulative cycle count were 
essential for the predictions. Accurate RUL prediction using the Random Forest model allows 
for proactive maintenance, cost savings, improved safety, and reduced environmental impact 
by maximizing battery life and preventing premature disposal. The findings highlight the 
potential of machine learning in optimizing battery management and advancing its 
applications across various industries. 
